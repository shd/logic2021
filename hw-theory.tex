\documentclass[10pt,a4paper,oneside]{article}
\usepackage[utf8]{inputenc}
\usepackage[english,russian]{babel}
\usepackage{amsmath}
\usepackage{amsthm}
\usepackage{amssymb}
\usepackage{enumerate}
\usepackage{stmaryrd}
\usepackage{cmll}
\usepackage{mathrsfs}
\usepackage[left=2cm,right=2cm,top=2cm,bottom=2cm,bindingoffset=0cm]{geometry}
\usepackage{proof}
\usepackage{tikz}
\usepackage{multicol}

\makeatletter
\newcommand{\dotminus}{\mathbin{\text{\@dotminus}}}

\newcommand{\@dotminus}{%
  \ooalign{\hidewidth\raise1ex\hbox{.}\hidewidth\cr$\m@th-$\cr}%
}
\makeatother

\usetikzlibrary{arrows,backgrounds,patterns,matrix,shapes,fit,calc,shadows,plotmarks}

\newtheorem{definition}{Определение}
\begin{document}

\begin{center}{\Large\textsc{\textbf{Теоретические (``малые'') домашние задания}}}\\
             \it Математическая логика, ИТМО, М3235-М3239, весна 2021 года\end{center}

\section*{Задание №1. Знакомство с исчислением высказываний.}

В рамках данного задания мы рассматриваем классическое исчисление высказываний с классическим
множеством истинностных значений $\{\text{И}, \text{Л}\}$.

\begin{enumerate}

\item Будем говорить, что высказывание общезначимо, если выполнено при любой оценке.
    Высказывание выполнимо, если существует оценка, при которой оно истинно.
    Высказывание опровержимо, если существует оценка, при которой оно ложно.
    Высказывание невыполнимо, если нет оценки, при которой оно истинно.
     Укажите про каждое из следующих высказываний, общезначимо, выполнимо, опровержимо или невыполнимо ли оно:
\begin{enumerate}
\item $\neg A\vee\neg\neg A$
\item $(A\rightarrow\neg B)\vee(B\rightarrow\neg C)\vee(C\rightarrow\neg A)$
\item $(((P\rightarrow Q)\rightarrow P)\rightarrow P)$
\item $\neg A \with \neg \neg A$
\item $\neg (A \with \neg A)$
\item $A$
\item $A \rightarrow A$
\item $A \rightarrow \neg A$
\item $(A \rightarrow B) \vee (B \rightarrow A)$
\end{enumerate}

\item Простые доказательства. Рассмотрим доказательства в классическом исчислении
высказываний, здесь используются следующие десять схем аксиом:

\begin{tabular}{ll}
(1) & $\phi \rightarrow (\psi \rightarrow \phi)$ \\
(2) & $(\phi \rightarrow \psi) \rightarrow (\phi \rightarrow \psi \rightarrow \pi) \rightarrow (\phi \rightarrow \pi)$ \\
(3) & $\phi \rightarrow \psi \rightarrow \phi \with \psi$\\
(4) & $\phi \with \psi \rightarrow \phi$\\
(5) & $\phi \with \psi \rightarrow \psi$\\
(6) & $\phi \rightarrow \phi \vee \psi$\\
(7) & $\psi \rightarrow \phi \vee \psi$\\
(8) & $(\phi \rightarrow \pi) \rightarrow (\psi \rightarrow \pi) \rightarrow (\phi \vee \psi \rightarrow \pi)$\\
(9) & $(\phi \rightarrow \psi) \rightarrow (\phi \rightarrow \neg \psi) \rightarrow \neg \phi$\\
(10) & $\neg \neg \phi \rightarrow \phi$
\end{tabular}

Докажите:
\begin{enumerate}
\item $\vdash A \rightarrow A$
\item $\vdash (A \rightarrow A \rightarrow B) \rightarrow (A \rightarrow B)$
\item $\vdash \neg (A \with \neg A)$
\item $\vdash A \with B \rightarrow B \with A$
\item $\vdash A \rightarrow \neg \neg A$
\item $A \with \neg A \vdash B$
\end{enumerate}

\item Известна теорема о дедукции: $\Gamma, \alpha \vdash \beta$ тогда и только тогда, 
когда $\Gamma \vdash \alpha \rightarrow \beta$. Докажите с её использованием:
\begin{enumerate}
\item $\neg A, B \vdash \neg(A\& B)$
\item $A,\neg B \vdash \neg( A\& B)$
\item $\neg A,\neg B \vdash \neg( A\& B)$
\item $\neg A,\neg B \vdash \neg( A\vee B)$
\item $ A,\neg B \vdash \neg( A\rightarrow B)$
\item $\neg A, B \vdash  A\rightarrow B$
\item $\neg A,\neg B \vdash  A\rightarrow B$
\item $\vdash (A \rightarrow B) \rightarrow (B \rightarrow C) \rightarrow (A \rightarrow C)$
\item $\vdash (A \rightarrow B) \rightarrow (B \rightarrow C) \rightarrow (C \rightarrow A)$
\item Закон контрапозиции: $\vdash (A \rightarrow B) \rightarrow (\neg B \rightarrow \neg A)$
\end{enumerate}

\item Докажите:
\begin{enumerate}
\item $\vdash A \vee \neg A$ \emph{(правило исключённого третьего)}
\item $\vdash A \with B \rightarrow \neg (\neg A \vee \neg B)$
\item $\vdash \neg (\neg A \with \neg B) \rightarrow A \vee B$
\item $\vdash A \with B \rightarrow A \vee B$
\item $\vdash ((A \rightarrow B) \rightarrow A)\rightarrow A$ \emph{(закон Пирса)}
\end{enumerate}

\item Даны высказывания $\alpha$ и $\beta$, причём $\vdash \alpha\rightarrow\beta$ и $\alpha\not\equiv\beta$. 
Укажите способ построения высказывания $\gamma$, такого, что
$\vdash\alpha\rightarrow\gamma$ и $\vdash\gamma\rightarrow\beta$, причём $\alpha\not\equiv\gamma$ и
$\beta\not\equiv\gamma$.

\item Покажите, что если $\alpha \vdash \beta$ и $\neg\alpha\vdash\beta$, то $\vdash\beta$.
\end{enumerate}

\section*{Задание №2. Теоремы о полноте и корректности классической логики, интуиционистская логика.}

\begin{enumerate}
\item Покажите, что если $\Gamma \vdash \alpha$, то $\Gamma \models \alpha$.

\item Покажите, что если $\Gamma \models \alpha$, то $\Gamma \vdash \alpha$.

\item \emph{О законе исключённого третьего.} Покажите, что в интуиционистском исчислении высказываний 
доказуемо следующее:

\begin{enumerate}
\item $((A\rightarrow B)\rightarrow A)\rightarrow A \vdash \neg\neg A \rightarrow A$
\item $A \vee \neg A \vdash \neg\neg A \rightarrow A$
\end{enumerate}

\item Предложим следующий способ оценки интуиционистских высказываний.
Фиксируем некоторое топологическое пространство с носителем $X$ и топологией
(множеством всех открытых множеств) $\Omega\subseteq\wp(X)$.
Множеством истинностных значений выберем $\Omega$.
Соответственно, функция оценок для переменных задаётся как $f_\mathcal{P}: \mathcal{P}\rightarrow\Omega$. 
Определим функции оценок для связок так:
$$\begin{array}{rcl}
f_\rightarrow(a,b) &:=& ((X\setminus a) \cup b)^\circ\\
f_\with(a,b) &:=& a \cap b\\
f_\vee(a,b) &:=& a \cup b\\
f_\neg(a) &:=& (X \setminus a)^\circ
\end{array}$$
Будем считать высказывание истинным, если его оценка --- всё пространство $X$.
Например, при $X = \mathbb{R}$ и $A := (0,\infty), B := (-\infty,1)$ высказывание $A\vee B$
истинно, но при $A := (0,\infty), B := (-\infty,0)$ оно ложно.

Известно, что интуиционистское исчисление высказываний корректно и полно при таком способе
оценки --- в частности это значит, что если формула $\alpha$ недоказуема, 
то найдётся такое топологическое пространство $X$
и такие оценки для пропозициональных переменных, что $\llbracket\alpha\rrbracket \neq X$.
Это позволяет показывать недоказуемость высказываний. Например, $\not\vdash A \vee \neg A$:
возьмём $X = \mathbb{R}$ и $A := (0,\infty)$. Тогда $\llbracket \neg A \rrbracket = (-\infty,0)$
и $\llbracket A \vee \neg A\rrbracket = \mathbb{R}\setminus\{0\} \ne \mathbb{R}$.

Предложите топологические пространства и оценку для пропозициональных переменных,
опровергающие следующие выскзывания:

\begin{enumerate}
\item $\neg A \vee \neg\neg A $
\item $(((A \rightarrow B) \rightarrow A) \rightarrow A)$
\item $\neg\neg A \rightarrow A$
\item $(A \rightarrow (B \vee \neg B)) \vee (\neg A \rightarrow (B \vee \neg B))$
\item $(A \rightarrow B) \vee (B \rightarrow C) \vee (C \rightarrow A)$
\end{enumerate}

\item Можно ли, имея $(A \rightarrow B) \vee (B \rightarrow C) \vee (C \rightarrow A)$, доказать
закон исключённого третьего в интуиционистской логике?

\item Известно, что в классической логике любая связка может быть \emph{выражена} как композиция 
конъюнкций и отрицаний: существует схема высказываний, использующая только конъюнкции и отрицания, 
задающая высказывание, логически эквивалентное исходной связке. 
Например, для импликации можно взять $\neg(\alpha\with\neg\beta)$, ведь 
$\alpha\rightarrow\beta\vdash\neg(\alpha\with\neg\beta)$ и $\neg(\alpha\with\neg\beta)\vdash\alpha\rightarrow\beta$. 
Возможно ли в интуиционистской логике выразить через остальные связки:
\begin{enumerate}
\item конъюнкцию?
\item дизъюнкцию?
\item импликацию?
\item отрицание?
\end{enumerate}
Если да, предложите формулу и два вывода. Если нет --- докажите это.

\item Назовём теорию \emph{противоречивой}, если в ней найдётся такое $\alpha$, что $\vdash\alpha$ и $\vdash\neg\alpha$.
Покажите, что исчисления высказываний (классическое и интуиционистское) противоречивы тогда и только тогда, 
когда в них доказуема любая формула. 

\item \emph{Теорема Гливенко.} Обозначим доказуемость высказывания $\alpha$ в классической логике 
как $\vdash_\text{к}\alpha$, а в интуиционистской --- как $\vdash_\text{и}\alpha$. 
Оказывается возможным показать, что какое бы ни было $\alpha$, если $\vdash_\text{к}\alpha$, 
то $\vdash_\text{и}\neg\neg\alpha$. А именно, покажите, что:

\begin{enumerate}
\item Если $\alpha$ --- аксиома, полученная из схем 1--9 исчисления высказываний, то $\vdash_\text{и}\neg\neg\alpha$.
\item $\vdash_\text{и}\neg\neg(\neg\neg\alpha\rightarrow\alpha)$
\item $\neg\neg\alpha,\neg\neg(\alpha\rightarrow\beta) \vdash_\text{и}\neg\neg\beta$
\item Докажите утверждение теоремы ($\vdash_\text{к}\alpha$ влечёт $\vdash_\text{и}\neg\neg\alpha$),
опираясь на предыдущие пункты, и покажите, что классическое исчисление высказываний противоречиво
тогда и только тогда, когда противоречиво интуиционистское.
\end{enumerate}

\end{enumerate}

\section*{Задание №3. Интуиционистская логика и натуральный вывод.}
\begin{enumerate}
\item Обозначим выводимость в ИИВ <<гильбертовского стиля>> как $\vdash_\text{и}$, 
а выводимость в ИИВ <<системы натурального (естественного) вывода>> как $\vdash_\text{е}$.

Заметим, что хоть языки этих исчислений и отличаются, мы можем построить преобразование 
высказываний этих исчислений друг в друга: приняв $\bot \Rightarrow A\with\neg A$ и $\neg \alpha \Rightarrow (\alpha\rightarrow\bot)$.
Будем обозначать высказывания в гильбертовском ИИВ обычными греческими буквами,
а соответствующие им высказывания в ИИВ натурального вывода --- 
буквами с апострофами: $\alpha', \beta', \dots$.

\begin{enumerate}
\item Пусть $\Gamma\vdash_\text{и}\alpha$. Покажите, что $\Gamma\vdash_\text{е}\alpha'$: предложите общую схему 
перестроения доказательства, постройте доказательства для одного случая базы и одного случая перехода индукции.
\item Пусть $\Gamma\vdash_\text{е}\alpha'$. Покажите, что $\Gamma\vdash_\text{и}\alpha$. 
\end{enumerate}

\item Рассмотрим $\mathbb{N}_0$ (натуральные числа с нулём) с традиционным отношением порядка как решётку.
Каков будет смысл операций $(+)$ и $(\cdot)$ в данной решётке, есть ли в ней псевдодополнение, 
определены ли 0 или 1? Приведите несколько свойств традиционных определений $(+)$ и $(\cdot)$, 
которые будут всё равно выполнены при таком переопределении, и несколько свойств, которые перестанут выполняться.

\item Постройте следующие примеры:
\begin{enumerate}
\item непустого частично-упорядоченного множества, не имеющего операций $(+)$ и $(\cdot)$ ни для каких элементов;
имеющего операцию $(+)$ для всех элементов, но не имеющего $(\cdot)$ для некоторых;
имеющего опреацию $(\cdot)$ для всех элементов, но не имеющего $(+)$ для некоторых.
\item решётки, не являющейся дистрибутивной решёткой;
 дистрибутивной, но не импликативной решётки; импликативной решётки без 0.
\end{enumerate}

\item Покажите следующие тождества и свойства для импликативных решёток:
\begin{enumerate}
\item ассоциативность: $a + (b + c) = (a + b) + c$ и $a \cdot (b \cdot c) = (a \cdot b) \cdot c$;
\item монотонность: пусть $a \preceq b$ и $c \preceq d$, тогда $a + c \preceq b + d$ и $a \cdot c \preceq b \cdot d$;
\item \emph{Законы поглощения:} $a \cdot (a + b) = a$; $a + (a \cdot b) = a$;
\item $a \preceq b$ выполнено тогда и только тогда, когда $a \rightarrow b = 1$;
\item из $a \preceq b$ следует $b\rightarrow c \preceq a\rightarrow c$ и $c\rightarrow a \preceq c \rightarrow b$;
\item из $a \preceq b \rightarrow c$ следует $a \cdot b \preceq c$;
\item $b \preceq a \rightarrow b$ и $a \rightarrow (b \rightarrow a) = 1$;
\item $a \rightarrow b \preceq ((a \rightarrow (b \rightarrow c)) \rightarrow (a \rightarrow c))$;
\item $a \preceq b \rightarrow a \cdot b$ и $a \rightarrow (b \rightarrow (a \cdot b)) = 1$
\item $a \rightarrow c \preceq (b \rightarrow c) \rightarrow (a + b \rightarrow c)$
\end{enumerate}

\item Покажите, что импликативная решётка дистрибутивна.
\item Покажите, что в дистрибутивной решётке (всегда $(a + b)\cdot c = (a \cdot c) + (b \cdot c)$) также выполнено
и $a + (b \cdot c) = (a + b) \cdot (a + c)$.

\item Рассмотрим топологическое пространство $\langle X, \Omega \rangle$, упорядочим его топологию $\Omega$ отношением $\subseteq$. 
Покажите, что такая конструкция является псевдобулевой алгеброй, а если топология --- дискретная (любое подмножество $X$ открыто),
то булевой алгеброй. 

\item Докажите, что ИИВ корректно, если в качестве модели выбрать псевдобулеву алгебру, а функции оценок определить так:
$$\begin{array}{ccc}
  \llbracket\alpha \with \beta\rrbracket & = & \llbracket\alpha\rrbracket \cdot \llbracket\beta\rrbracket \\
  \llbracket\alpha \vee \beta\rrbracket & = & \llbracket\alpha\rrbracket + \llbracket\beta\rrbracket \\
  \llbracket\alpha \rightarrow \beta\rrbracket & = & \llbracket\alpha\rrbracket \rightarrow \llbracket\beta\rrbracket \\
  \llbracket\neg\alpha\rrbracket & = & \llbracket\alpha\rrbracket \rightarrow 0 \\
  \llbracket\bot\rrbracket & = & 0 \\
\end{array}$$

\item Пусть задано отношение \emph{предпорядка} $R$ (транзитивное и рефлексивное, но необязательно антисимметричное) на множестве $A$.
Напомним несколько определений:
\begin{itemize}
\item определим отношение $R^= := \{ \langle x,y \rangle\ |\ xRy \text{ и } yRx \}$;
\item $[a]_{R^=} := \{ x\ |\ aR^=x \}$ --- класс эквивалентности, порождённый элементом $a$;
\item фактор-множество $A/{R^=} := \{ [a]_{R^=}\ |\ a \in A \}$;
\item на $A/{R^=}$ можно перенести отношение $R^* := \{ \langle [a],[b] \rangle\ |\ aRb \}$.
\end{itemize}

Покажите, что: отношение $R^=$ --- отношение эквивалентности; если $x \in [a]_{R^=}$, $y \in [b]_{R^=}$ и $aRb$, то $xRy$; отношение $R^*$ ---
отношение порядка на $A/{R^=}$.

\item Покажем, что конструкция из определения алгебры Линденбаума действительно является решёткой:
\begin{enumerate}
\item Покажите, что отношение $(\approx)$ --- отношение эквивалентности (напомним, что $\alpha\preceq\beta$, если $\alpha\vdash\beta$, а $\alpha\approx\beta$, если
$\alpha\vdash\beta$ и $\beta\vdash\alpha$). \emph{Подсказка:} воспользуйтесь предыдущим заданием.
\item Покажите, что $[\alpha]_\approx\cdot[\beta]_\approx = [\alpha\with\beta]_\approx$. Для этого, например, можно показать: 
\begin{itemize}
\item $\alpha\with\beta \preceq \alpha$;
\item если $\gamma \preceq \alpha$ и $\gamma\preceq\beta$, то $\gamma\preceq\alpha\with\beta$;
\item операция однозначно определена для всех элементов решётки (то есть определена для всех классов 
эквивалентности и не зависит от выбора представителей). \emph{Подсказка:} воспользуйтесь предыдущим заданием.
\end{itemize}
\item Покажите, что $[\alpha]+[\beta]=[\alpha\vee\beta]$. 
\item Покажите, что $[\alpha]\rightarrow[\beta]=[\alpha\rightarrow\beta]$.
\item Найдите классы эквивалентности для 0 и 1.
\end{enumerate}

\end{enumerate}
\end{document}
