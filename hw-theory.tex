\documentclass[10pt,a4paper,oneside]{article}
\usepackage[utf8]{inputenc}
\usepackage[english,russian]{babel}
\usepackage{amsmath}
\usepackage{amsthm}
\usepackage{amssymb}
\usepackage{enumerate}
\usepackage{stmaryrd}
\usepackage{cmll}
\usepackage{mathrsfs}
\usepackage[left=2cm,right=2cm,top=2cm,bottom=2cm,bindingoffset=0cm]{geometry}
\usepackage{proof}
\usepackage{tikz}
\usepackage{multicol}

\makeatletter
\newcommand{\dotminus}{\mathbin{\text{\@dotminus}}}

\newcommand{\@dotminus}{%
  \ooalign{\hidewidth\raise1ex\hbox{.}\hidewidth\cr$\m@th-$\cr}%
}
\makeatother

\usetikzlibrary{arrows,backgrounds,patterns,matrix,shapes,fit,calc,shadows,plotmarks}

\newtheorem{definition}{Определение}
\begin{document}

\begin{center}{\Large\textsc{\textbf{Теоретические (``малые'') домашние задания}}}\\
             \it Математическая логика, ИТМО, М3235-М3239, весна 2021 года\end{center}

\section*{Задание №1: <<знакомство с исчислением высказываний>>}

В рамках данного задания мы рассматриваем классическое исчисление высказываний с классическим
множеством истинностных значений $\{\text{И}, \text{Л}\}$.

\begin{enumerate}

\item Будем говорить, что высказывание общезначимо, если выполнено при любой оценке.
    Высказывание выполнимо, если существует оценка, при которой оно истинно.
    Высказывание опровержимо, если существует оценка, при которой оно ложно.
    Высказывание невыполнимо, если нет оценки, при которой оно истинно.
     Укажите про каждое из следующих высказываний, общезначимо, выполнимо, опровержимо или невыполнимо ли оно:
\begin{enumerate}
\item $\neg A\vee\neg\neg A$
\item $(A\rightarrow\neg B)\vee(B\rightarrow\neg C)\vee(C\rightarrow\neg A)$
\item $(((P\rightarrow Q)\rightarrow P)\rightarrow P)$
\item $\neg A \with \neg \neg A$
\item $\neg (A \with \neg A)$
\item $A$
\item $A \rightarrow A$
\item $A \rightarrow \neg A$
\item $(A \rightarrow B) \vee (B \rightarrow A)$
\end{enumerate}

\item Простые доказательства. Рассмотрим доказательства в классическом исчислении
высказываний, здесь используются следующие десять схем аксиом:

\begin{tabular}{ll}
(1) & $\phi \rightarrow (\psi \rightarrow \phi)$ \\
(2) & $(\phi \rightarrow \psi) \rightarrow (\phi \rightarrow \psi \rightarrow \pi) \rightarrow (\phi \rightarrow \pi)$ \\
(3) & $\phi \rightarrow \psi \rightarrow \phi \with \psi$\\
(4) & $\phi \with \psi \rightarrow \phi$\\
(5) & $\phi \with \psi \rightarrow \psi$\\
(6) & $\phi \rightarrow \phi \vee \psi$\\
(7) & $\psi \rightarrow \phi \vee \psi$\\
(8) & $(\phi \rightarrow \pi) \rightarrow (\psi \rightarrow \pi) \rightarrow (\phi \vee \psi \rightarrow \pi)$\\
(9) & $(\phi \rightarrow \psi) \rightarrow (\phi \rightarrow \neg \psi) \rightarrow \neg \phi$\\
(10) & $\neg \neg \phi \rightarrow \phi$
\end{tabular}

Докажите:
\begin{enumerate}
\item $\vdash A \rightarrow A$
\item $\vdash (A \rightarrow A \rightarrow B) \rightarrow (A \rightarrow B)$
\item $\vdash \neg (A \with \neg A)$
\item $\vdash A \with B \rightarrow B \with A$
\item $\vdash A \rightarrow \neg \neg A$
\item $A \with \neg A \vdash B$
\end{enumerate}

\item Известна теорема о дедукции: $\Gamma, \alpha \vdash \beta$ тогда и только тогда, 
когда $\Gamma \vdash \alpha \rightarrow \beta$. Докажите с её использованием:
\begin{enumerate}
\item $\neg A, B \vdash \neg(A\& B)$
\item $A,\neg B \vdash \neg( A\& B)$
\item $\neg A,\neg B \vdash \neg( A\& B)$
\item $\neg A,\neg B \vdash \neg( A\vee B)$
\item $ A,\neg B \vdash \neg( A\rightarrow B)$
\item $\neg A, B \vdash  A\rightarrow B$
\item $\neg A,\neg B \vdash  A\rightarrow B$
\item $\vdash (A \rightarrow B) \rightarrow (B \rightarrow C) \rightarrow (A \rightarrow C)$
\item $\vdash (A \rightarrow B) \rightarrow (B \rightarrow C) \rightarrow (C \rightarrow A)$
\item Закон контрапозиции: $\vdash (A \rightarrow B) \rightarrow (\neg B \rightarrow \neg A)$
\end{enumerate}

\item Докажите:
\begin{enumerate}
\item $\vdash A \vee \neg A$ \emph{(правило исключённого третьего)}
\item $\vdash A \with B \rightarrow \neg (\neg A \vee \neg B)$
\item $\vdash \neg (\neg A \with \neg B) \rightarrow A \vee B$
\item $\vdash A \with B \rightarrow A \vee B$
\item $\vdash ((A \rightarrow B) \rightarrow A)\rightarrow A$ \emph{(закон Пирса)}
\end{enumerate}

\item Даны высказывания $\alpha$ и $\beta$, причём $\vdash \alpha\rightarrow\beta$ и $\alpha\not\equiv\beta$. 
Укажите способ построения высказывания $\gamma$, такого, что
$\vdash\alpha\rightarrow\gamma$ и $\vdash\gamma\rightarrow\beta$, причём $\alpha\not\equiv\gamma$ и
$\beta\not\equiv\gamma$.

\item Покажите, что если $\alpha \vdash \beta$ и $\neg\alpha\vdash\beta$, то $\vdash\beta$.

\end{enumerate}
\end{document}
